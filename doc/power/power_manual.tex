\documentclass[letterpaper,twoside,10pt]{article}

\usepackage[letterpaper, margin=1in]{geometry}

\usepackage[USenglish]{babel} %francais, polish, spanish, ...
\usepackage[T1]{fontenc}
\usepackage[ansinew]{inputenc}
\usepackage{lmodern} %Type1-font for non-english texts and characters


\usepackage{graphicx} %%For loading graphic files
\usepackage{amsmath, xfrac, array, mdwlist, ulem}
\usepackage{amsthm}
\usepackage{amsfonts}
\usepackage{fancyvrb}
\usepackage[svgnames]{xcolor}
%\usepackage{natbib}
\usepackage[style=verbose, backend=bibtex]{biblatex}



\usepackage{hyperref}



\setlength{\parskip}{7pt}%
\setlength{\parindent}{0pt}%


\definecolor{LightGray}{gray}{0.9}

\bibliography{power_manual}


\makeatletter
\newif\ifFV@bgcolor
\newbox\FV@bgbox
\define@key{FV}{bgcolor}{\FV@bgcolortrue\def\FV@bgcolor{#1}}
\define@key{FV}{framecolor}{\FV@bgcolortrue\def\FV@framecolor{#1}}
\def\FV@framecolor{white}

\def\FV@BeginVBox{%
  \leavevmode\ifFV@bgcolor\setbox\FV@bgbox=\fi
  \hbox\ifx\FV@boxwidth\relax\else to\FV@boxwidth\fi\bgroup
  \ifcase\FV@baseline\vbox\or\vtop\or$\vcenter\fi\bgroup}
\def\FV@EndVBox{\egroup\ifmmode$\fi\hfil\egroup
  \ifFV@bgcolor\fcolorbox{\FV@framecolor}{\FV@bgcolor}{\box\FV@bgbox}\fi}
\makeatother

%\def\algorithmautorefname{Algorithm}


\begin{document}

\renewcommand*{\chapterautorefname}{Chapter}
\renewcommand*{\sectionautorefname}{Section}
\renewcommand*{\subsectionautorefname}{Section}
\renewcommand*{\subsubsectionautorefname}{Section}

\title{Power Estimation for VTR}
\author{Jeffrey Goeders}
%\date{} %%If commented, the current date is used.
\maketitle

\setcounter{tocdepth}{2}
\tableofcontents %Table of contents


\newpage
\section{Overview}
	TBD (\autoref{fig:flow}).
\begin{figure}[ht]
	\centering
	\includegraphics[scale=0.8]{images/flow.pdf}
	\caption{Power Estimation in the VTR Flow}
	\label{fig:flow}
\end{figure}


\newpage
\section{Running VTR with Power Estimation}
\subsection{VTR Flow}
\subsection{VTR Tasks}
\subsection{VPR 6.0}
The following command-line options are added to the VPR executable to facilitate power estimation (all are required): \newline
\texttt{\--\--power}:  Use this option to enable power estimation. \newline
\texttt{\--\--activity\_file <activities.act>}: The activity file, produce by ACE 2.0, or another tool. \newline
\texttt{\--\--tech\_properties <tech\_properties.xml>}: The technology properties XML file.


\newpage
\section{Supporting Tools}
\subsection{Technology Properties Generation}
Power estimation requires information detailing the properties of the CMOS technology.  
This information, which includes transistor capacitances, leakage currents, etc. is included in an \texttt{.xml} file, and provided as a parameter to VPR.
This XML file is generated using a script which automatically runs HSPICE, performs multiple circuit simulations, and extract the necessary values.


Some of these technology XML files are included with the release, and are located here: 

\begin{BVerbatim}[bgcolor=LightGray, boxwidth=\textwidth]
<vtr>/vtr_flow/tech/*
\end{BVerbatim}
			
If the user wishes to use a different CMOS technology file, they must run the following script.  HSPICE must be included in the \$PATH\$.

\begin{BVerbatim}[bgcolor=LightGray, boxwidth=\textwidth] 
<vtr>/vtr_flow/scripts/generate_cmos_tech_data.pl <tech_file> <tech_size> <Vdd> <temp>
\end{BVerbatim}

\texttt{<tech\_file>}: A SPICE technology file, containing a \texttt{pmos} and \texttt{nmos} models. \newline
\texttt{<tech\_size>}: The technology size, in meters. For example, a 90nm technology would have the value \texttt{90e-9}. \newline
\texttt{<Vdd>}: Supply voltage in Volts. \newline
\texttt{<temp>}: Operating temperature, in Celcius. 

\subsection{ACE 2.0 Activity Estimation}
Power estimation requires activity information for the entire netlist.  This ativity information consists of two values:
\vspace{-10pt}
\begin{enumerate}
	\item  \underline{The Signal Probability}, $P_1$, is the long-term probability that a signal is logic-high.  For example, a clock signal with a 50\% duty cycle will have $P_1(clk) = 0.5$.
	\item \underline{The Transition Density} (or switching activity), $A_S$, is the average number of times the signal will switch during each clock cycle.  For example, a clock has $A_S(clk)=2$.
\end{enumerate}

The default tool used to perform activity estimation in VTR is ACE 2.0 \footnote{\cite{ace2}}.  
This tool was originally designed to work with the Berkeley SIS tool, which is now obsolte.  ACE 2.0 was modifed to use ABC, and is included in the VTR package here:

\begin{BVerbatim}[bgcolor=LightGray, boxwidth=\textwidth] 
<vtr>/ace2
\end{BVerbatim}

The tool can be run using the following command-line arguments:

\begin{BVerbatim}[bgcolor=LightGray, boxwidth=\textwidth] 
<vtr>/ace2/ace -b <abc.blif> -o <activities.act> -n <new.blif>
\end{BVerbatim}

\texttt{<abc.blif>}: The input BLIF file produced by ABC. \newline
\texttt{<activities.act>}: The activity file to be created. \newline
\texttt{<new.blif>}: The new BLIF file.  This will be identical in function to the ABC blif; however, since ABC does not maintain internal node names, a new BLIF must be produced with node names that match the activity file. \newline
	
User's may with to use their own activity estimation tool.  The produced activity file must contain one line for each net in the BLIF file, in the following format: 

\begin{BVerbatim}[bgcolor=LightGray, boxwidth=\textwidth] 
<net name> <signal probability> <transistion density>
\end{BVerbatim}
	

\newpage
\section{Architecture Modelling}
The following section describes the architectural assumptions made by the power model, and the related parameters in the architecture file.

\subsection{Complex Blocks} \label{sec:clbs}
The VTR architecture description language supports a hierarchichal description of blocks.  In the architecture file, each block is described as a \texttt{pb\_type}, which may includes one or more children of type \texttt{pb\_type}, and interconnect structures to connect them.  The power estimation algorithm traverses this hierarchy recursively, and performs power estimation for each \texttt{pb\_type}.  The power model supports multiple power estimation methods, and the user specifies the desired method in the architecture file: \newline
\begin{BVerbatim}[bgcolor=LightGray, boxwidth=\textwidth] 
<pb_type>
	<power method="<est-method>">
</pb_type>
\end{BVerbatim}

The following is a list of valid estimation methods. Detailed descriptions of each type are provided in the following sections.  The methods are listed in order from most accurate to least accurate. 
\begin{enumerate}
\item \texttt{specify-size}: Detailed transistor level modelleling. The user supplies all buffer sizes and wire-lengths.  Any not provided by the user are ignored. 
\item \texttt{auto-size}: Detailed transistor level modelleling. The user can supply buffer sizes and wire-lengths; however, they will be automatically inserted when not provided. 
\item \texttt{pin-toggle}: Higher-level modelling.  The user specifies energy per toggle of the pins.  Static power provided as an absolute.
\item \texttt{C-internal}: Higher-level modelling.  The user supplies the internal capacitance of the block.  Static power provided as an absolute.
\item \texttt{absolute}: Highest-level modelling.  The user supplies both dynamic and static power as absolutes.
\end{enumerate}

If no estimation method is provided, it is inherited from the parent \texttt{pb\_type}. \textit{\textbf{If the top-level \texttt{pb\_type} has no estimation method, \texttt{auto-size} is assumed.}}  

\subsubsection{\texttt{specify-size}}
This estimation method provides a detailed transistor level modelling of CLBs, and will provide the most accurate power estimations.  For each \texttt{pb\_type}, power estimation accounts for the following components (See \autoref{fig:sample_clb}).
\begin{itemize*}
	\item Interconnect multiplexers
	\item Buffers and wire capacitances
	\item Child \texttt{pb\_types}
\end{itemize*}

\begin{figure}[ht]
	\centering
		\includegraphics[scale=0.5]{images/sample_clb.pdf}
	\caption{Sample Block}
	\label{fig:sample_clb}
\end{figure}


\textbf{Multiplexers} Interconnect multiplexers are modelled as 2-level pass-transistor multiplexers, comprised of minimum-size NMOS transistors. Their size is determined automatically from the \texttt{<interconnect/>} structures in the architecture description file.  

\textbf{Buffers and Wires} Buffers and wire capacitances are not defined in the architecture file, and must be explicitly added by the user.  They are assigned on a per port basis using the following construct:

\begin{BVerbatim}[bgcolor=LightGray, boxwidth=\textwidth] 
<pb_type>
	<input name="my_input" num_pins="1">
		<power ...options.../>
	</input>
</pb_type>
\end{BVerbatim}

The wire and buffer attributes can be set using the following options.  If no options are set, it is assumed that the wire capacitance is zero, and there are no buffers present.  Keep in mind that the port construct allows for multiple pins per port.  These attributes will be applied to each pin in the port.  If necessary, the user can seperate a port into multiple ports with different wire/buffer properties.
\begin{itemize}
	\item \texttt{wire\_capacitance="1.0e-15"}: The absolute capacitance of the wire, in Farads.
	\item \texttt{wire\_length="1.0e-7"}: The absolute length of the wire, in meters.  The local interconnect capacitance option must be specified, as described in \autoref{sec:local_interc_cap}.
	\item \texttt{wire\_length="auto"}: The wirelength is automatically sized.  See \autoref{sec:local_wire_autosize}.
	\item \texttt{buffer\_size="2.0"}: The size of the buffer at this pin.  See \autoref{sec:buffer_sizing} for more information.
	\item \texttt{buffer\_size="auto"}: The size of the buffer is automatically sized, assuming it drives the above wire capacitance and a single multiplexer.  
	See \autoref{sec:buffer_sizing} for more information.
\end{itemize}

\textbf{Primitives}
For all child \texttt{pb\_types}, the algorithm performs a recursive call. Eventually \texttt{pb\_types} will be reached that have no children.  
These are primitives, such as flip-flops, LUTs, or other hard-blocks.  The power model includes functions to perform transistor-level power estimation for flip-flops and LUTs. 
If the user wishes to use a design with other primitive types (memories, multipliers, etc), they must provide an equivalent function.  If the user makes such a function, the \texttt{power\_calc\_primitive} function should be modified to call it.  Alternatively, these blocks can be configured to use higher-level power estimation methods.

\textbf{NOTE:} Default Leakage Mode TBD

\subsubsection{\texttt{auto-size}}
This estimation method also performs detailed transistor-level modelling.  It is almost identical to the \texttt{specify-size} method described above.  The only difference is that the local wire capacitance and buffers are automatically inserted for all pins, when necessary.  This is equivalent to using the \texttt{specify-size} method with the \texttt{wire\_length="auto"} and 
\texttt{buffer\_size="auto"} options for every port.

\textbf{This is the default power estimation method.}  Although not as accurate as user-provided buffer and wire sizes, it is capable of automatically capturing trends in power dissipation as architectures are modified.  

\subsubsection{\texttt{pin-toggle}}
This method allows users to specify the dynamic power of a block in terms of the energy per toggle of each input, output or clock pin for the \texttt{pb\_type}.  The static power is provided as an absolute (in Watts). This is done using the following construct:

\begin{BVerbatim}[bgcolor=LightGray, boxwidth=\textwidth] 
<pb_type>
	<input name="my_input" num_pins="1">
		<power energy_per_toggle="1.0e-16">
	</input>
	...
	<power method="pin-toggle">
		<static_power power_per_instance="1.0e-6"/>
	</power>
</pb_type>
\end{BVerbatim}

This method does not perform any transistor-level estimations; the entire power estimation is performed using the above values.  \texttt{energy\_per\_toggle} specifies the energy consumed (in Joules) for a single toggle of a pin.  Keep in mind that the port construct allows for multiple pins per port.  These energy per toggle will be applied to each pin in the port.  If necessary, the user can seperate a port into multiple ports with different energy per toggle values.

\subsubsection{\texttt{C-internal}}
This method allows the users to specify the dynamic power of a block in terms of the internal capacitance of the block.  The activity will be averaged across all of the input pins, and will be supplied with the internal capacitance to the standard equation $P_{dyn}=\sfrac{1}{2}\alpha CV^2$.  Again, the static power is provided as an absolute (in Watts).  This is done using the following construct:

\begin{BVerbatim}[bgcolor=LightGray, boxwidth=\textwidth] 
<pb_type>
	<power method="c-internal">
		<dynamic_power C_internal="1.0e-16"/>
		<static_power power_per_instance="1.0e-6"/>
	</power>
</pb_type>
\end{BVerbatim}

\subsubsection{\texttt{absolute}}
This method is the most basic power estimation method, and allows users to specify both the dynamic and static power of a block as absolute values (in Watts).  This is done using the following construct:

\begin{BVerbatim}[bgcolor=LightGray, boxwidth=\textwidth] 
<pb_type>
	<power method="absolute">
		<dynamic_power power_per_instance="1.0e-6"/>
		<static_power power_per_instance="1.0e-6"/>
	</power>
</pb_type>
\end{BVerbatim}


\subsection{Global Routing}	
Global routing consists of switch boxes and input connection boxes.

\subsubsection{Switch Boxes} \label{sec:sb}
Switch boxes are modelled as the following components (\autoref{fig:sb}):
\vspace{-10pt}
\begin{enumerate*}
	\item Multiplexer 
	\item Buffer
	\item Wire capacitance
\end{enumerate*}


\begin{figure}[ht]
	\centering
		\includegraphics[scale=0.7]{images/sb.pdf}
	\caption{Switch Box}
	\label{fig:sb}
\end{figure}

\textbf{Multiplexer} 
	The multiplexer is modelled as 2-level pass-transistor multiplexer, comprised of minimum-size NMOS transistors.  
	The number of inputs to the multiplexer is automatically determined.
	
\textbf{Buffer} 
The buffer is a multistage CMOS buffer.  
Typically, this buffer is sized to drive the capacitance of the wire, as well as fanouts to other switch boxes and conneciton boxes.  The size of the buffer is specified by the user in the architecture file, as shown below. 

\begin{BVerbatim}[bgcolor=LightGray, boxwidth=\textwidth] 
<switchlist>
	<switch type="mux" ... power_buf_size="16.0"
</switchlist>
\end{BVerbatim}


Alternatively, the size can be replaced by the "\texttt{auto}" keyword, and the buffer will be automatically sized based on the load capacitance.  See \autoref{sec:buffer_sizing} for information about buffer sizings. 

\textbf{Wire Capacitance} The wire capacitance is determined using the segment information in the architecture file:

\begin{BVerbatim}[bgcolor=LightGray, boxwidth=\textwidth] 
<segmentlist>
	<segment type="unidir" ... power_options .../>
</segmentlist>
\end{BVerbatim}

The following options can be used to specify wire capacitance:
\vspace{-10pt}
\begin{itemize*}
\item \texttt{power\_wire\_c}: Absolute wire capacitance, in farads.
\item \texttt{power\_wire\_c\_per\_m}: Wire capacitance in farads per meter.  The capacitance is calculated using an automatically determined wirelength, based on the area of a tile in the FPGA (\autoref{sec:area_est}).
\end{itemize*}

\subsection{Input Connection Boxes} \label{sec:cb}
Input connection boxes are modelled as the following components (\autoref{fig:cb}):
\vspace{-10pt}
\begin{itemize}
\item One buffer per routing track, sized to drive the load of all input multiplexers to which the buffer is connected (For buffer sizing see \autoref{sec:buffer_sizing}).
\item One multiplexer per block input pin, sized according to the number of routing tracks that connect to the pin.
\end{itemize}

\begin{figure}[ht]
	\centering
		\includegraphics[scale=1.6]{images/cb.pdf}
	\caption{Conneciton Box}
	\label{fig:cb}
\end{figure}


\subsection{Clock Network}
The clock network modelled is a four quadrant spine and rib design, as illustrated in \autoref{fig:clock_network}. At this time, the power model only supports a single clock.
The model assumes that the entire spine and rib clock network will contain buffers separated in distance by the length of a grid tile.  
The buffer sizes and wire capacitances are specified in the architecture file using the following construct:

\begin{BVerbatim}[bgcolor=LightGray, boxwidth=\textwidth] 
<clocks>
	<clock ... clock_options .../>
</clocks>
\end{BVerbatim}


The following clock options are supported:
\vspace{-10pt}
\begin{itemize}
\item \texttt{C\_wire="1e-16"}: The absolute capacitance, in fards, of the wire between each clock buffer.
\item \texttt{C\_wire\_per\_m="1e-12"}: The wire capacitance, in fards per m.  The capacitance is calculated using an automatically determined wirelength, based on the area of a tile in the FPGA.
\item \texttt{buffer\_size="2.0"}: The size of each clock buffer.  This can be replaced with the "auto" keyword.  See \autoref{sec:buffer_sizing} for more information on buffer sizing.
\end{itemize}

\begin{figure}[ht]
	\centering
	\includegraphics[scale=0.7]{images/clock_network}
	\caption{The clock network.  Squares represent CLBs, and the wires represent the clock network.}
	\label{fig:clock_network}
\end{figure}




\clearpage
\section{Other Architecture Options \& Techniques}
\subsection{Area Estimations} \label{sec:area_est}
The power estimator performs area estimations of both the global routing structures and the complex blocks.  The CLB area is determined through a recursive function that counts the area of interconnect, buffers, and primitives (flip-flops and LUTs).  

If an architecture uses new primitives in CLBs, it should include a function that returns the transistor count.  This function should be called from within \texttt{power\_count\_transistors\_primitive()}.

\subsection{Local Wire Auto-Sizing} \label{sec:local_wire_autosize}
The power estimator contains a basic algorithm to estimate local interconnect wirelengths.  The algorithms determines the area of the interconnect structures, and calculates the side-length of the interconnect based on a square layout.  This side length is referred to as $L_{interc}$, as illustrated in \autoref{fig:local_interc_wirelength}.


\begin{figure}[ht]
	\centering
		\includegraphics[scale=0.7]{images/local_interc_wirelength.pdf}
	\caption{Local interconnect wirelength}
	\label{fig:local_interc_wirelength}
\end{figure}

It is assumed that each wire connecting a pin of a \texttt{pb\_type} to an interconnect structure is of length $0.5 \cdot L_{interc}$.  In reality, this length depends on the actual transistor
layout, and may be much larger or much smaller than the estimated value.  If desired, the user can override the 0.5 constant in the architecture file:

\begin{BVerbatim}[bgcolor=LightGray, boxwidth=\textwidth] 
<architecture>
	<power>
		<local_interconnect factor="0.5">
	</power>
</architecture>
\end{BVerbatim}

\subsection{Buffer Sizing} \label{sec:buffer_sizing}
In the power estimator, a buffer size refers to the size of the final stage of multi-stage buffer (if small, only a single stage is used).  The specified size is the (\sfrac{W}{L}) of the NMOS transistor.  The PMOS transistor will automatically be sized larger.    
Generally, buffers are sized depending on the load capacitance, using the following equation (logical effort model):

\begin{align}
	\text{Buffer Size}=\frac{1}{f_{LE}}*\frac{C_{Load}}{C_{INV}}
	\label{eq:autosize}
\end{align}	
In this equation, $C_{INV}$ is the input capacitance of a minimum-sized inverter, and $f_{LE}$ is the logical effort factor, which by default is 4.  This is standard design practise for high-performance devices.  The logical effort factor can be modified in the architecture file: 


\begin{BVerbatim}[bgcolor=LightGray, boxwidth=\textwidth] 
<architecture>
	<power>
		<buffers logical_effort_factor="4">
	</power>
</architecture>
\end{BVerbatim}

The number of buffer stages is chosen such that each stage is close to $f_{LE}$ times larger than the previous stage.






\subsection{Local Interconnect Capacitance} \label{sec:local_interc_cap}
If using the \texttt{auto-size} or \texttt{wire-length} options (\autoref{sec:clbs}), the local interconnect capacitance must be specified.  This is specified in the units of Farads/meter.

\begin{BVerbatim}[bgcolor=LightGray, boxwidth=\textwidth] 
<architecture>
	<power>
		<local_interconnect C_wire="1.0e-15">
	</power>
</architecture>
\end{BVerbatim}


\subsection{SRAM}
TBD

\newpage
\section{Support}
For support, please check \url{http://code.google.com/p/vtr-verilog-to-routing/wiki/Power}, or email \href{mailto:vtr.power.estimation@gmail.com}{vtr.power.estimation@gmail.com}.

\end{document}

